% This file requires XeLaTeX or LuaLaTeX to compile
% Use: xelatex rapport_data_pool_ar.tex
% Or: lualatex rapport_data_pool_ar.tex

\documentclass[12pt,a4paper]{article}
\usepackage{fontspec}
\usepackage{polyglossia}
\setmainlanguage[numerals=western]{arabic}
\setotherlanguage{french}
\setotherlanguage{english}

\usepackage{geometry}
\usepackage{graphicx}
\usepackage{booktabs}
\usepackage{longtable}
\usepackage{xcolor}
\usepackage{hyperref}
\usepackage{fancyhdr}
\usepackage{enumitem}

\geometry{margin=2.5cm}
\hypersetup{colorlinks=true, linkcolor=blue, urlcolor=blue}

% Arabic font - using Windows pre-installed fonts

\setmainfont{Arial}
\newfontfamily\arabicfont[Script=Arabic]{Arial}
\newfontfamily\englishfont{Arial}
\newfontfamily\arabicfonttt[Script=Arabic]{Courier New}

\pagestyle{plain}

\title{
    \vspace{-1cm}
    \textbf{\Huge تقرير إنشاء مجموعة البيانات}\\[0.5cm]
    \Large التنبؤ بأداء طلاب الثانوي المغاربة\\[0.3cm]
    \large الثانوية التأهيلية - جميع المستويات (الجذع المشترك، الأولى باك، الثانية باك)
}

\author{
    \textbf{المشروع:} التنبؤ بأداء الطلاب\\
    \textbf{التاريخ:} 5 فبراير 2026
}

\date{}

\begin{document}


\maketitle
\thispagestyle{empty}

\vspace{1cm}

\begin{abstract}
يوثق هذا التقرير عملية إنشاء مجموعة بيانات اصطناعية شاملة للتنبؤ بأداء الطلاب المغاربة في السلك الثانوي التأهيلي. تحتوي مجموعة البيانات المُنشأة على \textbf{10,000 سجل} طالب مع أكثر من \textbf{250 متغيراً} تغطي جميع الجوانب المتعلقة بتحليل والتنبؤ بالنجاح الدراسي.
\end{abstract}

\newpage
\tableofcontents
\newpage

\section{مقدمة}

\subsection{سياق المشروع}
يهدف مشروع التنبؤ بأداء الطلاب إلى تطوير نماذج تعلم آلي قادرة على تحديد العوامل المؤثرة في النجاح الدراسي لطلاب الثانوي المغاربة والتنبؤ باحتمالية نجاحهم في امتحان الباكالوريا.

\subsection{أهداف مجموعة البيانات}
يستجيب إنشاء مجموعة البيانات هذه لعدة أهداف:
\begin{itemize}
    \item توفير قاعدة بيانات واقعية وممثلة للطلاب المغاربة
    \item تضمين تنوع في الملفات الاجتماعية والاقتصادية والأكاديمية
    \item تغطية جميع جهات المملكة المغربية الـ12
    \item دمج متغيرات تنبؤية ذات صلة
\end{itemize}

\section{وصف مجموعة البيانات}

\subsection{الخصائص العامة}

\begin{table}[h]
\centering
\begin{tabular}{rr}
\toprule
\textbf{القيمة} & \textbf{الخاصية} \\
\midrule
10,000 طالب & عدد السجلات \\
أكثر من 250 & عدد المتغيرات \\
CSV (UTF-8) & صيغة الملف \\
17 ميغابايت & حجم الملف \\
الفرنسية & لغة البيانات \\
12 جهة & التغطية الجغرافية \\
\bottomrule
\end{tabular}
\caption{الخصائص العامة لمجموعة البيانات}
\end{table}

\subsection{فئات المتغيرات}

\subsubsection{1. المعلومات الشخصية والديموغرافية (20 متغير)}
معرف الطالب، الاسم الشخصي والعائلي، الجنس، تاريخ الميلاد والعمر، رمز مسار، الجهة، الإقليم، الجماعة، المنطقة، العنوان، الهاتف، البريد الإلكتروني

\subsubsection{2. المعلومات الدراسية (15 متغير)}
اسم المؤسسة، نوع المؤسسة، القطاع، الأكاديمية، المديرية، المستوى، الشعبة والتخصص، القسم، السنة الدراسية

\subsubsection{3. المعلومات العائلية (25 متغير)}
وضعية الوالدين، معلومات الأب والأم، المستوى التعليمي، المهنة، القطاع، الدخل، معلومات الولي، الدخل العائلي، عدد الإخوة، الترتيب في الأسرة

\subsubsection{4. الوضعية الاجتماعية والاقتصادية (20 متغير)}
نوع السكن، وضعية الملكية، عدد الغرف، الخدمات، التجهيزات المتوفرة، الكتب، المسافة إلى المدرسة، المنح والمساعدات

\subsubsection{5. المعلومات الصحية (15 متغير)}
التأمين الصحي، الأمراض المزمنة، الإعاقة، الحالة الصحية العامة، مؤشر كتلة الجسم، الطول، الوزن

\subsubsection{6. الأداء الأكاديمي (أكثر من 50 متغير)}
درجات جميع المواد للدورة الأولى والثانية والمعدل السنوي: اللغة العربية، الفرنسية، الإنجليزية، الرياضيات، الفيزياء والكيمياء، علوم الحياة والأرض، التاريخ والجغرافيا، التربية الإسلامية، الفلسفة، التربية البدنية، المعلوميات، الاقتصاد، المحاسبة، التدبير

\subsubsection{7. عادات الدراسة (25 متغير)}
ساعات الدراسة اليومية، الدروس الخصوصية، تكلفة الدعم، مكان ووقت الدراسة، الدراسة الجماعية، الموارد الرقمية، نسبة إنجاز الواجبات

\subsubsection{8. الأنشطة الموازية (20 متغير)}
الأنشطة الرياضية، الأنشطة الفنية، الأندية والجمعيات، العمل التطوعي، المشاركة في الأحداث المدرسية

\subsubsection{9. التوجيه والطموحات (15 متغير)}
الطموح المهني، الجامعة المرغوبة، مجال الدراسة، مشروع الدراسة في الخارج، البلد المستهدف

\subsubsection{10. العوامل النفسية (20 متغير)}
مستوى التحفيز، الثقة بالنفس، مستوى التوتر والقلق، قلق الامتحانات، الرضا المدرسي، العلاقات مع الأقران، الدعم العائلي

\subsubsection{11. نمط الحياة (15 متغير)}
التغذية، ساعات النوم، وقت النوم والاستيقاظ، النشاط البدني، وقت الشاشة، الشبكات الاجتماعية، القراءة

\subsubsection{12. المهارات والقدرات (20 متغير)}
فعالية التعلم الذاتي، تحديد الأهداف، إدارة الوقت، التنظيم، حل المشكلات، التفكير النقدي، التواصل، أسلوب التعلم

\subsubsection{13. المهارات اللغوية (10 متغير)}
مستويات اللغات: العربية، الفرنسية، الإنجليزية، الإسبانية، الألمانية، الناطق بالأمازيغية


\section{التغطية الجغرافية}

تغطي مجموعة البيانات الجهات الإدارية الـ12 للمغرب:
\begin{enumerate}
    \item الرباط-سلا-القنيطرة
    \item الدار البيضاء-سطات
    \item فاس-مكناس
    \item مراكش-آسفي
    \item طنجة-تطوان-الحسيمة
    \item سوس-ماسة
    \item الشرق
    \item بني ملال-خنيفرة
    \item درعة-تافيلالت
    \item العيون-الساقية الحمراء
    \item كلميم-واد نون
    \item الداخلة-وادي الذهب
\end{enumerate}

\section{الشعب المغطاة}

\begin{table}[h]
\centering
\begin{tabular}{rr}
\toprule
\textbf{التخصصات} & \textbf{الشعبة} \\
\midrule
العلوم الفيزيائية، علوم الحياة والأرض & العلوم التجريبية \\
رياضيات أ، رياضيات ب & العلوم الرياضية \\
التدبير المحاسباتي، العلوم الاقتصادية & العلوم الاقتصادية \\
الآداب، العلوم الإنسانية & الآداب والعلوم الإنسانية \\
علوم ميكانيكية، علوم كهربائية & العلوم والتكنولوجيات \\
\bottomrule
\end{tabular}
\caption{شعب وتخصصات الباكالوريا المغطاة}
\end{table}

\section{منهجية التوليد}

\subsection{النهج المستخدم}
تم توليد البيانات بشكل اصطناعي باستخدام Python وفق المبادئ التالية:
\begin{itemize}
    \item \textbf{الواقعية}: القيم مبنية على توزيعات واقعية تتوافق مع السياق المغربي
    \item \textbf{الاتساق}: احترام الارتباطات المنطقية بين المتغيرات
    \item \textbf{التنوع}: مجموعة واسعة من الملفات الاجتماعية والاقتصادية
    \item \textbf{التمثيلية}: توزيع متناسب بين المناطق الحضرية وشبه الحضرية والقروية
\end{itemize}



\section{تحليل وتجهيز البيانات}

قبل البدء في عملية استكشاف البيانات، تم إجراء مرحلة أولية للتحليل والتنظيف لضمان جودة البيانات.


\subsection{التحليل الأولي}
أظهر الفحص الأولي لمجموعة البيانات الخصائص التالية:
\begin{itemize}
    \item \textbf{الحجم}: 10,000 سجل و286 متغير.
    \item \textbf{أنواع البيانات}: 65 متغير عشري (float64)، 48 متغير صحيح (int64)، و173 متغير نصي (object).
    \item \textbf{القيم المفقودة}: تم تحديد أعمدة فارغة تماماً (100\% قيم مفقودة) مثل \textenglish{type\_handicap, economie\_s1, remarques}، وأخرى بقيم مفقودة جزئياً مثل \textenglish{etablissement\_precedent} (5925 قيمة مفقودة) و \textenglish{annees\_redoublees} (4537 قيمة مفقودة).
\end{itemize}


\subsection{إجراءات التنظيف}
تم تطبيق الخطوات التالية لتجهيز الملف النهائي : \\ \textenglish{\texttt{Morocco\_Student\_Data\_Cleaned.csv}}
\begin{enumerate}
    \item \textbf{حذف الأعمدة الفارغة}: إزالة 17 عموداً لا تحتوي على أي بيانات.
    \item \textbf{معالجة القيم المفقودة}: تعويض القيم المفقودة في سنوات التكرار بصفر، والأنشطة بـ "لا يوجد".
    \item \textbf{تنسيق البيانات}: توحيد تنسيق التواريخ لتواريخ الميلاد والجمع.
\end{enumerate}

\section{استخدام مجموعة البيانات}

\subsection{التطبيقات الممكنة}
\begin{itemize}
    \item تطوير نماذج التنبؤ بالنجاح الدراسي
    \item تحديد عوامل خطر الرسوب
    \item تحليل اللامساواة التعليمية
    \item تصميم أنظمة الإنذار المبكر
\end{itemize}

\subsection{المتغير المستهدف}
المتغير المستهدف للتنبؤ هو \textenglish{\textbf{moyenne\_annuelle}}: المعدل العام للطالب في السنة الدراسية الحالية. هذا مشروع \textbf{انحدار} \textenglish{(Regression)}.

\subsection{النماذج الموصى بها}
\begin{itemize}
    \item \textbf{الانحدار الخطي}: نموذج أساسي للمرجعية
    \item \textenglish{\textbf{Random Forest Regressor}}: للعلاقات غير الخطية
    \item \textenglish{\textbf{XGBoost / Gradient Boosting}}: أداء عالي للبيانات الجدولية
    \item \textbf{التعلم العميق} \textenglish{(MLP)}: للعلاقات المعقدة
\end{itemize}

\section{خاتمة}

تشكل مجموعة البيانات هذه مورداً شاملاً وواقعياً لتطوير نماذج التنبؤ بأداء الطلاب المغاربة. مع 10,000 سجل وأكثر من 250 متغير، توفر قاعدة صلبة للتحليل الاستكشافي وهندسة الخصائص وتدريب نماذج التعلم الآلي.

\vspace{1cm}

\noindent\textbf{الملفات المُنشأة:}\\
\textenglish{\texttt{Morocco\_Student\_Data\_Pool.csv}} (البيانات الخام)\\
\textenglish{\texttt{Morocco\_Student\_Data\_Cleaned.csv}} (البيانات المنظفة)\\
\textbf{تاريخ الإنشاء:} 5 فبراير 2026\\
\textbf{الترميز:} UTF-8

\end{document}
