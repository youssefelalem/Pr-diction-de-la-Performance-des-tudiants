\documentclass[12pt,a4paper]{article}
\usepackage[utf8]{inputenc}
\usepackage[T1]{fontenc}
\usepackage[french]{babel}
\usepackage{geometry}
\usepackage{graphicx}
\usepackage{booktabs}
\usepackage{longtable}
\usepackage{xcolor}
\usepackage{hyperref}
\usepackage{fancyhdr}
\usepackage{titlesec}
\usepackage{enumitem}

\geometry{margin=2.5cm}
\hypersetup{colorlinks=true, linkcolor=blue, urlcolor=blue}

\pagestyle{fancy}
\fancyhf{}
\fancyhead[L]{\textit{Rapport - Creation du Data Pool}}
\fancyhead[R]{\textit{Prediction Performance Etudiants}}
\fancyfoot[C]{\thepage}

\titleformat{\section}{\Large\bfseries\color{blue!70!black}}{\thesection}{1em}{}
\titleformat{\subsection}{\large\bfseries\color{blue!50!black}}{\thesubsection}{1em}{}

\title{
    \vspace{-1cm}
    \textbf{\Huge Rapport de Creation du Data Pool}\\[0.5cm]
    \Large Prediction de la Performance des Etudiants Marocains\\[0.3cm]
    \large Lycees Qualifiants - Tous les Niveaux (Tronc Commun, 1BAC, 2BAC)
}

\author{
    \textbf{Projet:} Prediction de la Performance des Etudiants\\
    \textbf{Date:} 5 Fevrier 2026
}

\date{}

\begin{document}

\maketitle
\thispagestyle{empty}

\vspace{1cm}

\begin{abstract}
Ce rapport documente le processus de creation d'un ensemble de donnees synthetiques complet pour la prediction de la performance des etudiants marocains du cycle secondaire qualifiant. Le dataset genere contient \textbf{10 000 enregistrements} d'etudiants avec plus de \textbf{250 variables (features)} couvrant tous les aspects pertinents pour l'analyse et la prediction de la reussite scolaire.
\end{abstract}

\newpage
\tableofcontents
\newpage

\section{Introduction}

\subsection{Contexte du Projet}
Le projet de prediction de la performance des etudiants vise a developper des modeles de machine learning capables d'identifier les facteurs influencant la reussite scolaire des lyceens marocains et de predire leur probabilite de reussite au baccalaureat.

\subsection{Objectif du Data Pool}
La creation de ce data pool repond a plusieurs objectifs:
\begin{itemize}[leftmargin=*]
    \item Fournir une base de donnees realiste et representative de la population estudiantine marocaine
    \item Inclure une diversite de profils socio-economiques et academiques
    \item Couvrir toutes les regions du Maroc
    \item Integrer des variables predictives pertinentes basees sur la litterature scientifique
\end{itemize}

\section{Description du Dataset}

\subsection{Caracteristiques Generales}

\begin{table}[h]
\centering
\begin{tabular}{ll}
\toprule
\textbf{Caracteristique} & \textbf{Valeur} \\
\midrule
Nombre d'enregistrements & 10 000 etudiants \\
Nombre de variables & 250+ features \\
Format du fichier & CSV (UTF-8) \\
Taille du fichier & $\sim$17 MB \\
Langue des donnees & Francais \\
Couverture geographique & 12 regions du Maroc \\
\bottomrule
\end{tabular}
\caption{Caracteristiques generales du dataset}
\end{table}

\subsection{Categories de Variables}

Le dataset est organise en plusieurs categories thematiques:

\subsubsection{1. Informations Personnelles et Demographiques (20 variables)}
\begin{itemize}
    \item Identifiant etudiant (id\_etudiant)
    \item Prenom, Nom, Nom complet
    \item Sexe (M/F)
    \item Date de naissance, Age
    \item Code Massar (identifiant national)
    \item Region, Province, Commune
    \item Zone (Urbain/Semi-Urbain/Rural)
    \item Adresse, Code postal
    \item Telephone, Email
\end{itemize}

\subsubsection{2. Informations Scolaires (15 variables)}
\begin{itemize}
    \item Nom de l'etablissement
    \item Type d'etablissement (Lycee Qualifiant)
    \item Secteur (Public/Prive)
    \item Academie regionale
    \item Direction provinciale
    \item Niveau (Tronc Commun / 1BAC / 2BAC)
    \item Filiere et Specialite
    \item Classe
    \item Annee d'inscription et annee scolaire
\end{itemize}

\subsubsection{3. Informations Familiales (25 variables)}
\begin{itemize}
    \item Statut des parents (vivant/decede)
    \item Informations sur le pere: prenom, nom, niveau d'education, profession, secteur d'activite, revenu
    \item Informations sur la mere: prenom, nom, niveau d'education, profession, secteur d'activite, revenu
    \item Information sur le tuteur
    \item Revenu familial total et source de revenu
    \item Nombre de freres et soeurs, rang dans la fratrie
    \item Statut parental (Maries/Divorces/Veuf)
\end{itemize}

\subsubsection{4. Conditions Socio-economiques (20 variables)}
\begin{itemize}
    \item Type de logement (Villa/Appartement/Maison/Maison Traditionnelle)
    \item Statut de propriete (Proprietaire/Locataire)
    \item Nombre de pieces
    \item Acces aux services (electricite, eau, internet)
    \item Equipements disponibles (chambre personnelle, bureau, ordinateur, laptop, tablette, smartphone)
    \item Livres a la maison
    \item Distance a l'ecole et moyen de transport
    \item Bourses et aides (Tayssir, bourses sociales, bourses d'excellence)
\end{itemize}

\subsubsection{5. Informations de Sante (15 variables)}
\begin{itemize}
    \item Assurance maladie et type
    \item Maladies chroniques
    \item Handicap
    \item Port de lunettes, problemes auditifs, allergies
    \item Etat de sante general
    \item IMC, taille, poids
\end{itemize}

\subsubsection{6. Performance Academique (50+ variables)}

Notes pour chaque matiere (Semestre 1, Semestre 2, Moyenne annuelle):
\begin{itemize}
    \item Arabe
    \item Francais
    \item Anglais
    \item Mathematiques
    \item Physique-Chimie
    \item Sciences de la Vie et de la Terre (SVT)
    \item Histoire-Geographie
    \item Education Islamique
    \item Philosophie
    \item Education Physique
    \item Informatique
    \item Economie, Comptabilite, Gestion (pour les filieres economiques)
\end{itemize}

Indicateurs globaux:
\begin{itemize}
    \item Moyenne generale (S1, S2, Annuelle)
    \item Rang dans la classe
    \item Effectif de la classe
    \item Absences (totales, justifiees, non justifiees)
    \item Retards
    \item Avertissements et sanctions
    \item Comportement
\end{itemize}

\subsubsection{7. Habitudes d'Etude (25 variables)}
\begin{itemize}
    \item Heures d'etude par jour et weekend
    \item Cours particuliers et matieres
    \item Cout du soutien scolaire
    \item Lieu et moment prefere d'etude
    \item Etude en groupe
    \item Utilisation des ressources en ligne
    \item Taux de remise des devoirs
    \item Participation en classe
    \item Prise de notes
    \item Utilisation de la bibliotheque
\end{itemize}

\subsubsection{8. Activites Parascolaires (20 variables)}
\begin{itemize}
    \item Activites sportives et type de sport
    \item Activites artistiques
    \item Clubs et associations
    \item Benevolat
    \item Participation aux evenements scolaires
    \item Membre du conseil des eleves
\end{itemize}

\subsubsection{9. Orientation et Aspirations (15 variables)}
\begin{itemize}
    \item Aspiration de carriere
    \item Universite souhaitee
    \item Domaine d'etude souhaite
    \item Projet d'etudes a l'etranger
    \item Pays cible
    \item Connaissance de l'orientation
    \item Seances d'orientation suivies
\end{itemize}

\subsubsection{10. Facteurs Psychologiques (20 variables)}
\begin{itemize}
    \item Niveau de motivation
    \item Confiance en soi
    \item Niveau de stress et d'anxiete
    \item Anxiete aux examens
    \item Satisfaction scolaire
    \item Relations avec les pairs
    \item Soutien familial
    \item Implication parentale
    \item Attentes et pression parentales
\end{itemize}

\subsubsection{11. Mode de Vie (15 variables)}
\begin{itemize}
    \item Petit dejeuner et repas par jour
    \item Heures de sommeil (semaine/weekend)
    \item Heure de coucher et de reveil
    \item Activite physique
    \item Temps d'ecran
    \item Reseaux sociaux
    \item Jeux video
    \item Lecture
    \item Consommation de cafeine
    \item Travail a temps partiel
\end{itemize}

\subsubsection{12. Competences et Aptitudes (20 variables)}
\begin{itemize}
    \item Efficacite d'auto-apprentissage
    \item Capacite a fixer des objectifs
    \item Gestion du temps
    \item Organisation
    \item Resolution de problemes
    \item Pensee critique
    \item Communication
    \item Style d'apprentissage (Visuel/Auditif/Kinesthesique)
    \item Intelligence dominante
    \item Creativite, Adaptabilite, Resilience
\end{itemize}

\subsubsection{13. Competences Linguistiques (10 variables)}
\begin{itemize}
    \item Niveau en arabe (C2-A1)
    \item Niveau en francais (C1-A2)
    \item Niveau en anglais (B2-A1)
    \item Niveau en espagnol
    \item Niveau en allemand
    \item Autres langues
    \item Locuteur amazigh
\end{itemize}


\section{Couverture Geographique}

Le dataset couvre les 12 regions administratives du Maroc:

\begin{enumerate}
    \item Rabat-Sale-Kenitra
    \item Casablanca-Settat
    \item Fes-Meknes
    \item Marrakech-Safi
    \item Tanger-Tetouan-Al Hoceima
    \item Souss-Massa
    \item Oriental
    \item Beni Mellal-Khenifra
    \item Draa-Tafilalet
    \item Laayoune-Sakia El Hamra
    \item Guelmim-Oued Noun
    \item Dakhla-Oued Ed-Dahab
\end{enumerate}

\section{Filieres Couvertes}

\begin{table}[h]
\centering
\begin{tabular}{ll}
\toprule
\textbf{Filiere} & \textbf{Specialites} \\
\midrule
Sciences Experimentales & Sciences Physiques (PC), SVT \\
Sciences Mathematiques & Maths A (SMA), Maths B (SMB) \\
Sciences Economiques & Gestion Comptable (SGC), Sciences Economiques (SE) \\
Lettres et Sciences Humaines & Lettres, Sciences Humaines \\
Sciences et Technologies & Sciences Mecaniques (STM), Sciences Electriques (STE) \\
\bottomrule
\end{tabular}
\caption{Filieres et specialites du baccalaureat couvertes}
\end{table}

\section{Methodologie de Generation}

\subsection{Approche Utilisee}
Les donnees ont ete generees de maniere synthetique en utilisant Python avec les principes suivants:

\begin{itemize}
    \item \textbf{Realisme}: Les valeurs sont basees sur des distributions realistes correspondant au contexte marocain
    \item \textbf{Coherence}: Les correlations logiques entre variables sont respectees (ex: revenu familial et equipements disponibles)
    \item \textbf{Diversite}: Large eventail de profils socio-economiques et academiques
    \item \textbf{Representativite}: Distribution proportionnelle entre zones urbaines, semi-urbaines et rurales
\end{itemize}



\section{Analyse et Nettoyage des Données}

Avant de procéder à l'exploration des données, une phase initiale d'analyse et de nettoyage a été réalisée pour garantir la qualité des données.


\subsection{Analyse Initiale}
L'examen initial du dataset a révélé les caractéristiques suivantes :
\begin{itemize}
    \item \textbf{Dimensions} : 10 000 enregistrements et 286 variables.
    \item \textbf{Types de données} : 65 variables flottantes (float64), 48 entiers (int64) et 173 objets (object).
    \item \textbf{Valeurs manquantes} : Identification de colonnes entièrement vides (100\% manquantes) telles que \texttt{type\_handicap, economie\_s1, remarques}, et d'autres partiellement renseignées comme \texttt{etablissement\_precedent} (5925 manquantes) et \texttt{annees\_redoublees} (4537 manquantes).
\end{itemize}


\subsection{Actions de Nettoyage}
Les étapes suivantes ont été appliquées pour produire le fichier final : \\ \texttt{Morocco\_Student\_Data\_Cleaned.csv}
\begin{enumerate}
    \item \textbf{Suppression des colonnes vides} : Retrait des 17 colonnes ne contenant aucune donnée.
    \item \textbf{Traitement des valeurs manquantes} : Imputation des années de redoublement par 0 et des activités par "Aucun".
    \item \textbf{Formatage des données} : Standardisation du format des dates de naissance et de collecte.
\end{enumerate}

\section{Utilisation du Dataset}

\subsection{Applications Possibles}
\begin{itemize}
    \item Developpement de modeles de prediction de reussite scolaire
    \item Identification des facteurs de risque d'echec
    \item Analyse des inegalites educatives
    \item Conception de systemes d'alerte precoce
    \item Etudes sur l'impact des facteurs socio-economiques
\end{itemize}

\subsection{Variable Cible}
La variable cible a predire est \textbf{moyenne\_annuelle}: le moyenne generale de l'etudiant pour l'annee scolaire en cours. Il s'agit d'un probleme de \textbf{regression}.

\subsection{Modeles Recommandes}
\begin{itemize}
    \item \textbf{Regression Lineaire}: Modele de base pour etablir une reference
    \item \textbf{Random Forest Regressor}: Capture les relations non-lineaires
    \item \textbf{XGBoost / Gradient Boosting}: Performance elevee pour les donnees tabulaires
    \item \textbf{Deep Learning (MLP)}: Reseaux de neurones pour les relations complexes
\end{itemize}

\section{Conclusion}

Ce data pool constitue une ressource complete et realiste pour le developpement de modeles de prediction de la performance des etudiants marocains. Avec ses 10 000 enregistrements et plus de 250 variables, il offre une base solide pour l'analyse exploratoire, le feature engineering, et l'entrainement de modeles de machine learning.

\vspace{1cm}

\noindent\textbf{Fichiers générés:}\\
\texttt{Morocco\_Student\_Data\_Pool.csv} (Données brutes)\\
\texttt{Morocco\_Student\_Data\_Cleaned.csv} (Données nettoyées)\\
\textbf{Date de creation:} 5 Fevrier 2026\\
\textbf{Encodage:} UTF-8

\end{document}
